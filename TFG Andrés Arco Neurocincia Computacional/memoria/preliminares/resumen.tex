\chapter*{Resumen}

% Los artículos y libros incluidos en el archivo research.bib pueden
% citarse desde cualquier punto del texto usando ~\cite.

Actualmente conocemos menos del diez por ciento del funcionamiento del cerebro. La Neurociencia Computacional permite construir modelos de simulación del cerebro, que incluyen cientos de miles de neuronas, y ayudar así a neurocientíficos y médicos a descifrar la funcionalidad del cerebro. 

En este proyecto se ha simulado un modelo de cerebro ampliamente usado para describir la dinámica neuronal del circuito cortical a nivel de micro- y meso-escala. A partir de la simulación de actividad neuronal, se ha generado la señal del electroencefalograma (EEG), una de las señales no-invasivas más conocida en el ámbito clínico. 

La pregunta científica que hemos abordado en este proyecto ha sido si podemos estimar los parámetros del modelo (por ejemplo, el cociente entre excitación, E, e inhibición, I: E/I) que han dado lugar a las distintas propiedades de la señal del EEG. Para ello, hemos desarrollado herramientas de machine learning (ML) que permiten encontrar automáticamente las regiones de interés del espacio de parámetros del modelo cortical y que se relacionan con cambios en la señal del EEG. Estas herramientas podrán usarse para ayudar en el diagnóstico clínico de condiciones médicas y descodificar los parámetros del circuito cortical que son alterados.

\textbf{Palabras clave:} modelo neuronal, encefalograma, EEG, neuronas excitadoras, neuronas inhibidoras, NEST, regresión lineal, Ridge, K-NNeighbors, red neuronal.