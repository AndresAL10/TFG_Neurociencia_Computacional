\chapter*{Abstract}

We currently know less than ten percent of how the brain works. Computational Neuroscience makes it possible to build simulation models of the brain, which include hundreds of thousands of neurons, that are very helpful for neuroscientists and doctors to understand the functionality of the brain.

In this project, a widely used brain model has been simulated to describe the neural dynamics of the cortical circuit at the micro- and meso-scale level. From the simulation of neuronal activity, the electroencephalogram (EEG) signal has been generated, one of the most well-known non-invasive signals in the clinical field.

The scientific question that we have addressed in this project has been whether we can estimate the parameters of the model (for example, the ratio between excitation, E, and inhibition, I: E/I) that have given rise to the different properties of the signal of the EEG. To do this, we have developed machine learning (ML) tools that automatically find the regions of interest in the parameter space of the cortical model and that are related to changes in the EEG signal. These tools may be used to aid in the clinical diagnosis of medical conditions and decode the cortical circuitry parameters that are altered.

\textbf{Keywords:} neuronal model, encephalogram, EEG, excitatory neurons, inhibitory neurons, NEST, linear regression, Ridge, K-NNeighbors, neural network.